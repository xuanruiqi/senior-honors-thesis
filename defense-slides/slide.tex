\documentclass[11pt]{beamer}
\usetheme{Warsaw}
\usepackage[utf8]{inputenc}
\usepackage[english]{babel}
\usepackage{amsmath}
\usepackage{amsfonts}
\usepackage{amssymb}
\usepackage{graphicx}
\usepackage{natbib}
\usepackage{xcolor}
\usepackage{listings}

\lstdefinestyle{base}{
  basicstyle=\ttfamily\color{black},
  moredelim=**[is][\color{red}]{@}{@},
  moredelim=**[is][\color{blue}]{**}{**}
}

\author{Xuanrui (Ray) Qi}
\title{Elephant Tracks II: Practical, Extensible Memory Tracing}
% \setbeamercovered{transparent} 
% \setbeamertemplate{navigation symbols}{} 
% \titlegraphic{\includegraphics[height=1.5cm]{tuftscs}} 
\institute{Department of Computer Science,\\Tufts University} 
\date{Senior Honors Thesis Defense\\May 3, 2018\\~\\
  \small
  Committee: Samuel Z. Guyer, Kathleen Fisher}
\begin{document}

\begin{frame}
  \titlepage
\end{frame}

%\begin{frame}
%\tableofcontents
%\end{frame}

\begin{frame}
  \frametitle{Goals of my thesis}
  \textbf{Make memory tracing great again!}
  \begin{enumerate}
  \item Make memory tracing fast(er)
  \item Make memory tracing extensible and eventually available for
    languages other than Java
  \item Develop new, language-agnostic techniques to run the Merlin algorithm
  \end{enumerate}
\end{frame}

\begin{frame}{What is memory tracing?}
  \begin{itemize}
  \item \textbf{Complete} chronological record of what happened in the heap
  \item Procedure entry \& exit, allocations, pointers updates/overwrites, deaths, etc.
  \end{itemize}
\end{frame}

\begin{frame}{Why do we need memory traces?}
  \begin{itemize}
  \item In a world where all memory is manually managed...
    \pause
  \item Do we know when memory is allocated? \pause
    \textit{Yes! We allocated that.} \pause
  \item Do we know when objects die? \pause
    \textit{Hopefully! If not, we will forget to deallocate them and create leaks.}
    \pause
  \item What happens if we reference a deallocated object or an invalid pointer 
    then? \pause \textit{We don't know exactly. But likely our program will blow up.}
  \end{itemize}
\end{frame}

\begin{frame}{Why do we need memory traces?}
  \begin{itemize}
  \item In a world where all memory is heap-allocated and automatically managed...
    \pause
  \item What happens if we reference a deallocated object or an invalid pointer
    address? \pause \textit{Nothing, because it can't happen. Yay!}\pause
  \item But do we know when memory is allocated? \pause
    \textit{Sometimes...}\pause
  \item Do we know when objects die? \pause \textit{Almost never...} \pause
  \item Do we want to know these details, then? \pause \textit{Yes!}
  \end{itemize}
  
\end{frame}

\begin{frame}{What can we use memory traces for?}
  \begin{itemize}
  \item Evaluating GC performance
  \item Help develop new GC algorithms
  \item Learn new facts about object lifetime patterns \citep{Garbology}
  \item Find memory leaks \citep{MemInsight}
  \item Help programmers understand their memory footprint
  \end{itemize}
\end{frame}

\begin{frame}{How to get memory traces?}
  \begin{itemize}
  \item Through dynamic analysis/instrumentation
    \pause
  \item Generating records on each allocation, pointer update, procedure entry/exit, etc.
    \pause
  \item What about deaths? They aren't obvious through dynamic analysis!
  \end{itemize}
\end{frame}

\begin{frame}{The solution}
  \begin{itemize}
  \item Compute them!
    \pause
  \item Compute each object's \textbf{idealized death time} using the Merlin algorithm 
    \citep{Merlin, MerlinSIGMETRICS}.
    \pause
  \item Idealized death time = latest time at which an object is shown possible to be reached.
  \end{itemize}
\end{frame}

\begin{frame}{Merlin}
  \begin{equation*}
    T_d(o) = \mathrm{max}(T_s(o), \{T_d(p) \mid \forall p : p \rightarrow o\})
  \end{equation*}
  $T_s(o)$: last-accessed timestamp of object\\
  $T_d(o)$: ``idealized death time'' of object\\~\\
  \pause
  The death time of an object is the max of:
  \begin{itemize}
  \item the last time it was accessed; and
  \item the death times of all objects pointing to it
  \end{itemize}
\end{frame}

\begin{frame}{Merlin: the algorithm}
  \begin{itemize}
  \item Use an iterative method to compute death times
  \item ``Propagate'' timestamps by depth-first search
  \item Example on the board
  \end{itemize}
\end{frame}

\begin{frame}{How do we trace program execution exactly?}
  \begin{itemize}
  \item \textbf{Instrument} the program
    \pause
  \item Insert code on-the-fly to generate runtime traces
    \pause
  \item Important: whenever an object is used, emit a ``witness'' record
    \pause
  \item Analyze traces to generate death records
   
  \end{itemize}
\end{frame}

\begin{frame}{Can't we already do this?}
  \begin{itemize}
  \item Yes, you're right. There's Elephant Tracks \citep{ElephantTracks}.
    \pause
  \item But it's slow, heavyweight, and not very portable...
    \pause
  \item And it only works for Java
    \pause
  \item That's why we need a new tool...
  \end{itemize}
\end{frame}

\begin{frame}{Elephant Tracks II architecture}
  \begin{itemize}
  \item Frontend: trace program, generate execution records, timestamp events, etc.
    \pause
  \item Backend: do trace analysis, compute death records
    \pause
  \item Pluggable architecture: one backend, multiple frontends
  \item Separates complex computations from slow tracing
  \end{itemize}
\end{frame}

\begin{frame}{Frontend implementation}
  \begin{itemize}
  \item Currently, only Java frontend implemented
  \item Uses \textbf{JVMTI} to instrument program bytecode \emph{at runtime}
  \item Uses \textbf{JNIF} \citep{JNIF} to manipulate bytecode
  \end{itemize}
\end{frame}

\begin{frame}{Event detection in ET2/Java}
  \begin{itemize}
  \item Event detection: finding events that need to be traced
    \pause
  \item ET2/Java trace generation is \emph{not completely dynamic}
    \pause
  \item Bytecode search for certain key instructions
  \end{itemize}
\end{frame}

\begin{frame}[fragile]{Instrumenting Java programs}
  \begin{lstlisting}[language=jvmis, style=base]
    52: aload_1
    @53: invokevirtual #9@
    56: goto          89
    59: aload_0
    @60: getfield      #3@
    63: ifnonnull     81
    66: aload_0
  \end{lstlisting}
\end{frame}

\begin{frame}[fragile]{Instrumenting Java programs}
  \begin{lstlisting}[language=jvmis, style=base]
    52: aload_1
    @53: invokevirtual #9@
    **##: invokestatic (method entry)**
    56: goto          89
    59: aload_0
    @60: getfield      #3@
    **##: invokestatic (witness)**
    63: ifnonnull     81
    66: aload_0
  \end{lstlisting}
\end{frame}

\begin{frame}{Main optimizations}
  \begin{itemize}
  \item \textbf{``Native'' bytecode manipulation}\\
    \textit{ET}: bytecode goes into a separate Java process, gets rewritten, and 
    then sent back (very slow)\\
    \textit{ET2}: bytecode gets rewritten directly in the JVMTI agent (much better)
    
    \pause
    
  \item \textbf{Java-based instrumentation}\\
    \textit{ET}: each instrumentation call is an FFI call to a C++ function (expensive)\\
    \textit{ET2}: everything happens in Java (leverages JIT)
  \end{itemize}
\end{frame}

\begin{frame}{The backend}
  \begin{itemize}
  \item Also called the \textbf{GC simulator}
  \item Workflow: ``execute'' the trace, run GC simulation as appropriate, compute death times after each GC
    \pause
  \item Few assumptions about the memory model!
    \pause
  \item Only assumptions: must have heap-allocated blocks, pointers, and GC
  \end{itemize}
\end{frame}

\begin{frame}{Simulating GC}
  \begin{itemize}
  \item Problem: no exact knowledge of GC roots inside the traces, so can't run GC
    \pause
  \item Solution: use a approximate, conservative strategy!
    \pause
  \item Treat everything possibly alive in the current context as GC roots
  \end{itemize}
\end{frame}

\begin{frame}{``Conservative root searching''}
  \begin{itemize}
  \item Keep a stack that simulates the call stack
    \pause
  \item Generate records for each parameter to procedure on entry
    \pause
  \item ``Parameter'' records and ``witness'' records pushed to the stack
    \pause
  \item When GC triggered, use everything on the stack as root
  \end{itemize}
\end{frame}

\begin{frame}{Extending ET2}
  \begin{itemize}
  \item ET2 is \textbf{extensible} by design
    \pause
  \item One backend, multiple languages, multiple frontends
    \pause
  \item Same simple execution model
  \end{itemize}
\end{frame}

\begin{frame}{Why extend ET2?}
  \begin{itemize}
  \item Support different languages
    \pause
  \item Study memory use in functional programming languages
    \begin{itemize}
    \item First-class functional closures
    \item Lazy FPLs \& thunks
    \end{itemize}
    \pause 
  \item Compare memory usage in different languages
  \end{itemize}
\end{frame}

\begin{frame}{Current state of ET2}
  \begin{itemize}
  \item ET2/Java is mostly usable
  \item Some edge cases unimplemented (e.g. reflection)
  \item Around 10-100 times faster than Elephant Tracks
  \item ET2 backend is still work in progress (outside collaboration)
  \end{itemize}
\end{frame}

\begin{frame}{In the thesis...}
  \begin{itemize}
  \item Details on the architecture \& trace format
  \item Details on the algorithms described here
  \item More about implementation specifics
  \end{itemize}
\end{frame}

\begin{frame}
  \frametitle{Thanks to my collaborators}
  \includegraphics[scale=0.1]{google.jpg}
  \hspace*{\fill}
  \includegraphics[scale=0.1]{anu.jpeg}\\
  
  \textbf{Google}: JC Beyler, Man Cao, Wessam Hassanein, Kathryn McKinley, Ryan Rose,
  Leandro Watanabe
  
  \textbf{ANU}: Steve Blackburn

  (all in alphabetical order)
\end{frame}

\begin{frame}
  \frametitle{and finally...}
  Thanks to my committee...
  \pause
  \newline
  and thanks to everyone who came today!
\end{frame}

\begin{frame}[allowframebreaks]{References}
  \def\newblock{}
  \bibliographystyle{acmref}
  \bibliography{slide}
\end{frame}

\end{document}
