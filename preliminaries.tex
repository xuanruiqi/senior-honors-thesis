\section{The fundamental problem of garbage collection analysis}
Memory management is difficult, and the horrors of manual memory management need
not be reiterated. Reference counting is a simple alternative to manual memory
management, but suffers from two major drawbacks: (1) it incurs significant
runtime cost, as the reference count needs to be updated whenever a new alias is
created or destroyed for each pointer, and (2) it results in memory leaks whenever
cycles are formed in the reference graph. Graph-based garbage collection algorithms
fix both problems, and are easy to parallelize and optimize \citep{GCHandbook}.

Although garbage collection is a much more efficient algorithm than reference counting
for most purposes, it has a few serious drawbacks. Garbage collection algorithms,
no matter how refined, cause random pauses during program execution which are unpredictable
and nondeterministic. Moreover, garbage collctors cannot tell us when objects die; all we could
know is that garbage-collected objects die \emph{some time} between the last garbage
collection run and the current garbage collection run.

Thus, the properties of garbage collectors impede the analysis and evaluation of themselves, and
external tools are often required to analyze garbage collectors. For example, a good garbage collector
should minimize the time span $\Delta t$ between an object's death time $t_{d}$
and collection time $t_{c}$. A reasonable metric that a garbage collector designer (and implementer)
would like to minimize might be the average ``time to collection'', i.e.
\begin{equation} \label{eq:1}
  \overline{\Delta t} = \frac{1}{n_{obj}}\sum_{i=1}^{n_{obj}}(t_c^i - t_d^i)
\end{equation}

However, without extra analysis tools, one could not obtain even a reasonable guess for the
$t_{d}$ value of any object, and therefore could not compute $\overline{\Delta t}$ using \ref{eq:1}.
Obtaining death times for each object is (in principle) easy: for example, one could simply do
``mock'' reference count on each object (without actually deallocating dead objects) and
recording the object death times. Nevertheless, this method, albeit intuitive, suffers from
the same drawbacks that plague the effectiveness of reference counting: this will degrade runtime
performance (such that the validity of timestamps obtained become questionable), and this method
could not generate death records for objects that form part of a cycle in the reference graph.

For these reasons, neither Elephant Tracks II nor other GC tracing tools \citep{MemInsight, ElephantTracks}
use reference counting to generate death record. Instead, we use the Merlin algorithm \citep{Merlin}, which
will be discussed hereafter, to \emph{compute} precise object death records using information gathered
at runtime.

\section{The Merlin algorithm}
Instead of (painstakingly) building a reference counter into a runtime system and generate incomplete death
records with it, one could compute the \emph{idealized death time} of objects using the Merlin algorithm, which
is based on a simple insight. There are two ways that object ``die'', i.e. become disconnected from the reference
graph: when there are no more references to the object, or when all objects that hold references to the said object
have died \citep{Merlin, MerlinSIGMETRICS, ElephantTracks}. Thus, we could compute the death time of each object recursively:
\newpage
\begin{lstlisting}[
    caption={Haskell-style pseudocode for the Merlin algorithm},
    label={lst:1}, basicstyle=\small,
    frame=single,
    numbers=none,
    language=haskell]
  deathTime :: Object -> Timestamp
  deathTime o = max (timeStamp o)
                    (maximum pointsToDeathTimes)
      where
      pointsToDeathTime = map deathTime (allPointsTo o)
\end{lstlisting}
where \lstinline{allPointsTo o} generates a list of objects that point to \lstinline{o}, and\\
\lstinline{timeStamp o} fetches the ``timestamp'' for \lstinline{o}. For the purposes of the Merlin algorithm,
the \emph{timestamp} for an object is the last time at which the object is found to be accessible or ``alive''.

The na\"{i}ve recursive version of the Merlin algorithm is illustrative for purposes of understanding the
algorithm, but is inefficient for actual implementation, and in fact, will not terminate when the reference graph
contains cycles. However, \cite{Merlin} also gives an efficient, stack-based, implementation, based
on depth-first search through the reference graph, that runs in $\Theta(n \log n)$ time and has good
experimental results.

Another way of thinking about the Merlin algorithm is that the idealized death time for each object, in a cyclic
graph, is that the idealized death time of each object is the least fixed-point that satisfies the equation
\begin{equation*}
  T_d(o) = \mathrm{max}(T_s(o), \{T_d(p) \mid \forall p : p \rightarrow o\})
\end{equation*}
where $T_s$ is the current timestamp function and $T_d$ is the deathtime function. The DFS-based implementation is simply
an instance of the iterative method for the least fixed-point under this definition.

Additionally, although the Merlin algorithm is designed to run on each garbage collection,
it could also be run offline if garbage collections are recorded in the GC trace. In Elephant Tracks, the online
approach is used; however, in Elephant Tracks II, the offline approach will be used.

With the Merlin algorithm, the problem of generating death records is transformed into one of finding timestamps
for each object, but the Merlin algorithm itself leaves this as an open question. For Java-like languages, the
best marker of an object being alive is probably the object being ``used'', e.g. an instance method called on the
object, the object being assigned to a pointer, etc. However, the strategy used is up to the implementer, and
the effectiveness of Merlin is up to effective timestamp generation for each object.

\section{Elephant Tracks}
Elephant Tracks \citep{ElephantTracks} is the major motivation for the work in this thesis, and represents a major attempt
to use the Merlin algorithm for practical GC tracing. Elephant Tracks is implemented as a JVM Tool Interface (JVMTI) agent, or a shared library
containing callbacks for the JVM to call on a number of events, e.g. whenever a class is loaded.

To generate timestamps for each object, Elephant Tracks rewrites the bytecode of each class before it is loaded, and inserts
calls to methods that record significant events. For example, whenever Elephant Tracks find an object allocation in a method,
it inserts a call to a method that records the allocation event. Whenever Elephant Tracks finds a \texttt{getfield} instruction
in a method, it inserts a call to update the timestamp for the object subject to the instruction, as a \texttt{getfield} on an
object indicates the aliveness of the object.

When a class is about to be loaded, the JVM calls the callback function supplied to it,
while the function sends the bytecode to a bytecode manipulator process. The ``bytecode manipulator'' rewrites the class using
the ASM bytecode instrumentation library, and sends the instrumented bytecode back after it is done with instrumentation. Then, at
runtime, the new, instrumented bytecode will be executed, and records will be generated by the inserted ``interceptor'' methods. As
all production-level JVMs provide implementations of the JVMTI, Elephant Tracks is, in principle, compatible with all standards-conformant
JVM implementations.

While Elephant Tracks is mostly successful in generating precise and complete GC traces for Java programs, one may easily see that
the implementation of Elephant Tracks is extremely inefficient. First of all, inter-process communication between the agent and the
``bytecode manipulator'' is required on \emph{every single} class load, which adds extremely high overhead to the running Java program.
Moreover, the ``interceptor'' methods are all native methods, requiring the JVM to invoke the JNI on any ``interesting'' event. Finally,
this complex architecture has rendered the code base for Elephant Tracks no longer maintainable and extremely prone to bit-rot. In fact,
by February 2018, Elephant Tracks can only run on a very old version of IBM J9 JVM, using outdated versions of the ASM library, and can
slow down program execution by 800 to 1000 times. Initially, Elephant Tracks was proposed as a fast and portable alternative to its alternatives
such as GCTrace, but due to architectural and design flaws, Elephant Tracks no longer meets its initial design goals.

Our new tool, Elephant Tracks II, is a tool ``in the spirit of'' Elephant Tracks, but is free of many design flaws of Elephant Tracks, making
it easily extensible and maintainable.
