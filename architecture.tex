\section{The architecture of Elephant Tracks}
\cite{ElephantTracks} listed six design goals of Elephant Tracks: precise, complete, informative,
well-defined, portable, and fast. However, the latest release of Elephant Tracks \citep{ElephantTracksPage}
barely achieves three of those goals, namely well-defined, portable and fast.

According to \cite{ElephantTracksPage}, the current release of Elephant Tracks fails to run on most Java
implementations (supporting only IBM J9), does not support recent Java implementations, and can be
irritatingly slow. Most of the problems bugging Elephant Tracks could be attributed to its architecture.

The old Elephant Tracks --- like the new Java frontend to Elephant Tracks --- is written as a JVM Tool Interface
agent \citep{ElephantTracks}. The JVMTI, according to Oracle's specifications, is a interface which should be implemented
by all standards-conforming JVMs that allow developers and debuggers to dynamically inspect and control the state
and execution of Java programs \citep{JVMTI} running in a JVM, in a fashion similar to the GNU Debugger
(\lstinline{gdb})'s state inspection and execution control facilities. For example, \lstinline{jdb}, the
\lstinline{gdb}-style debugger included in both of Oracle's Java implementations (Oracle JDK and OpenJDK), is implemented
as a JVMTI agent.

The JVMTI does not specify how the JVM shall interact with JVMTI agents besides stating that they should ``run in the same
process with and communicate directly with the virtual machine executing the application being examined''. However, since
JVMTI agents are compiled to native shared objects on the host system, most if not all JVM implementations dynamically
link with the JVMTI agent at runtime and call into the JVMTI agent at specific events. For example, a (rather primitive)
profiler might want to set a callback on the entry and exit of each method to generate a call graph. At run time, every time
a method is entered or exited, the JVM calls into the provided callback function and executes it. The JVMTI, therefore, is an
ideal technique to implement a memory tracer. Thus, when we designed the Java frontend for ET2, we chose to continue to use
the JVMTI as a portable and efficient technique to gather traces.

Elephant Tracks uses dynamic \emph{bytecode rewriting} as its major implementation technique. A JVMTI callback is set such that
Elephant Tracks gets access to a class's metadata and bytecode right before a class is loaded. Then, Elephant Tracks inspects
the bytecode of each method in the class, searching for important events such as object allocation (i.e. a \lstinline{new} instruction)
or a pointer update (usually a \lstinline{putfield} instruction). When an important event is found, the bytecode rewriter inserts a
call to a static method in a ``proxy'' class. After Elephant Tracks rewrite the class, it is loaded and executed. During
execution, the ``proxy'' method generates appropriate traces for the event it witnesses.

For example, let us consider the following Java method, taken from a real benchmark used to profile ET2:
\begin{lstlisting}[
    caption={Elephant Tracks example: source code},
    label={lst:1}, basicstyle=\small,
    frame=single,
    numbers=none,
    language=java]
  public class BinarySearchTree<T extends Comparable<T> > {
      // other methods omitted
    
      public void insert(T insertVal) {
          if (val == null) {
              val = insertVal;
              return;
          }

          if (val.compareTo(insertVal) <= 0) {
              if (right == null) {
                  right = new BinarySearchTree<T>(insertVal);
              } else {
                  right.insert(insertVal);
              }
          } else if (left == null) {
              left = new BinarySearchTree<T>(insertVal);
          } else {
              left.insert(insertVal);
          }
      }
  }
\end{lstlisting}

An analysis at the source code level would allow us to spot the allocation sites and pointer updates and
``liveness witnesses'' (i.e. instructions that proves the liveness of an object) in the code:
\begin{lstlisting}[
    caption={Elephant Tracks example: annotated source code},
    label={lst:1}, basicstyle=\small,
    frame=single,
    numbers=none,
    language=java]

  public class BinarySearchTree<T extends Comparable<T> > {
      // other methods omitted
      
      public void insert(T insertVal) {
          // witness the liveness of the callee of ``insert''
        
          if (val == null) {
              // pointer update: val -> insertVal
              val = insertVal;
              return;
          }

          if (val.compareTo(insertVal) <= 0) {
            if (right == null) {
                  // allocation and pointer update
                  right = new BinarySearchTree<T>(insertVal);
              } else {
                  right.insert(insertVal);
              }
          } else if (left == null) {
              // allocation and pointer update
              left = new BinarySearchTree<T>(insertVal);
          } else {
              left.insert(insertVal);
          }
      }
  }
\end{lstlisting}


Next, we examine the bytecode of the \lstinline{compile} method. The \lstinline{BinarySearchTree} class
is compiled using \lstinline{javac} version 1.8.0\_161 and then disassembled using \lstinline{javap}. We have
added comments that marks the places where Elephant Tracks would inject method calls.
\begin{lstlisting}[
    caption={Elephant Tracks example: annotated bytecode},
    label={lst:1}, basicstyle=\small,
    frame=single,
    numbers=none,
    language=jvmis]
  public void insert(T);
      Code:
         0: aload_0
         1: getfield      #2  // Field val:Ljava/lang/Comparable;
         4: ifnonnull     13
         7: aload_0
         8: aload_1
         // pointer update
         9: putfield      #2  // Field val:Ljava/lang/Comparable;
        // method exit
        12: return
        13: aload_0
        14: getfield      #2  // Field val:Ljava/lang/Comparable;
        17: aload_1
        // liveness witness (``val'') & method call
        18: invokeinterface #6,  2  // InterfaceMethod java/lang/Comparable.compareTo:(Ljava/lang/Object;)I
        23: ifgt          59
        26: aload_0
        27: getfield      #4  // Field right:LBinarySearchTree;
        30: ifnonnull     48
        33: aload_0
        // allocation
        34: new           #7  // class BinarySearchTree
        37: dup
        38: aload_1
        // method call
        39: invokespecial #8  // Method "<init>":(Ljava/lang/Comparable;)V
        42: putfield      #4  // Field right:LBinarySearchTree;
        45: goto          89
        48: aload_0
        49: getfield      #4  // Field right:LBinarySearchTree;
        52: aload_1
        // liveness witness (``right'') & method call
        53: invokevirtual #9  // Method insert:(Ljava/lang/Comparable;)V
        56: goto          89
        59: aload_0
        60: getfield      #3  // Field left:LBinarySearchTree;
        63: ifnonnull     81
        66: aload_0
        // allocation
        67: new           #7  // class BinarySearchTree
        70: dup
        71: aload_1
        // method call
        72: invokespecial #8  // Method "<init>":(Ljava/lang/Comparable;)V
        // pointer update
        75: putfield      #3  // Field left:LBinarySearchTree;
        78: goto          89
        81: aload_0
        82: getfield      #3  // Field left:LBinarySearchTree;
        85: aload_1
        // liveness witness (``left'') & method call
        86: invokevirtual #9  // Method insert:(Ljava/lang/Comparable;)V
        // method exit
        89: return
\end{lstlisting}

As one might see, there are two deciding factors that constrain Elephant Tracks' efficiency: the efficiency
of bytecode rewriting, and the number (and efficiency) of inserted method calls. However, these two aspects are exactly
the places where Elephant Tracks perform badly.

Elephant Tracks 
