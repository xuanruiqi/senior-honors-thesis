The red-black tree benchmark consists of three tasks: inserting one million numbers (the integers 1 to 1,000,000) into
an empty red-black tree, printing the height of the red-black tree, and then in-order traversing the red-black tree. The
implementation of the red-black tree is based on \cite{PFDS}'s purely functional implementation.

All benchmarks are run on a machine running Ubuntu Linux 16.04 LTS with 16 gigabytes of RAM and a quadcore Intel i7-6700HQ
CPU (base frequency 2.60 GHz).

The Haskell implementation of the benchmark consists of two files, ``RBT.hs'' (implements the red-black tree) and
``Bench.hs'' (runs the benchmark). The program is compiled using the Glasgow Haskell Compiler (GHC) version
8.0.2, with the ``-O2'' optimization flag enabled.

\begin{lstlisting}[
  caption={The red-black tree benchmark in Haskell: RBT.hs},
  language=haskell,
  basicstyle=\small,
  numbers=none,
  frame=single]
{-# LANGUAGE DeriveFoldable, DeriveFunctor, DeriveTraversable #-}

module RBT
  ( Tree(..)
  , member
  , insert
  , height
  , inOrder
  )
where

data Color = Red | Black
data Tree a = E
            | T Color (Tree a) a (Tree a)
            deriving (Functor, Traversable, Foldable)

member :: Ord a => a -> Tree a -> Bool
member _ E = False
member x (T _ l v r)
  | x < v  = member x l
  | x == v = True
  | x > v  = member x r

insert :: Ord a => a -> Tree a -> Tree a
insert x t = case ins t of
               T _ l v r -> T Black l v r
  where
    ins E = T Red E x E
    ins (T color l v r)
      | x < v     = balance color (ins l) v r
      | x > v     = balance color l v (ins r)
      | otherwise = T color l v r

balance :: Ord a => Color -> Tree a -> a -> Tree a -> Tree a
balance Black (T Red (T Red a x b) y c) z d = T Red (T Black a x b) y (T Black c z d)
balance Black (T Red a x (T Red b y c)) z d = T Red (T Black a x b) y (T Black c z d)
balance Black a x (T Red (T Red b y c) z d) = T Red (T Black a x b) y (T Black c z d)
balance Black a x (T Red b y (T Red c z d)) = T Red (T Black a x b) y (T Black c z d)
balance c l v r = T c l v r

height :: Ord a => Tree a -> Integer
height E = 0
height (T _ l _ r) = 1 + max (height l) (height r)

inOrder :: Ord a => Tree a -> [a]
inOrder E = []
inOrder (T _ l x r) = inOrder l ++ (x:inOrder r)
\end{lstlisting}

\begin{lstlisting}[
  caption={The red-black tree benchmark in Haskell: Bench.hs},
  language=haskell,
  basicstyle=\small,
  numbers=none,
  frame=single]
  module Main (main) where

  import qualified RBT

  main :: IO ()
  main = do
    let rbt = foldl (flip RBT.insert) RBT.E [1..1000000]
    putStr "Height of tree: "
    print (RBT.height rbt)
    mapM_ print (RBT.inOrder rbt)
\end{lstlisting}

\newpage
The Standard ML implementation of the benchmark consists of three files, ``rbt.sig'' (defines signatures),
``rbt.sml'' (implements the signatures), and ``bench.sml'' (runs the benchmark). The program is
compiled using the optimizing SML compiler MLton (build number 20180207).

\begin{lstlisting}[
  caption={The red-black tree benchmark in SML: rbt.sig},
  language=ml,
  basicstyle=\small,
  numbers=none,
  frame=single]
  signature ORDERED =
  sig
      type T
             
      val eq : (T * T) -> bool
      val lt : (T * T) -> bool
      val leq : (T * T) -> bool
  end

  signature BINARYTREE =
  sig
      structure Element : ORDERED
      type Elem = Element.T
                         
      type Tree

      val empty : Tree
      val member : (Elem * Tree) -> bool
      val height : Tree -> int
      val insert : (Elem * Tree) -> Tree
      val inOrder : Tree -> Elem list
  end

\end{lstlisting}

\begin{lstlisting}[
  caption={The red-black tree benchmark in SML: rbt.sml},
  language=ml,
  basicstyle=\small,
  numbers=none,
  frame=single]
  functor RedBlackTree (Value: ORDERED) : BINARYTREE =
  struct

  structure Element = Value
  type Elem = Element.T
                        
  datatype Color = RED | BLACK
  datatype Tree = E
                | T of Color * Tree * Elem * Tree


  val empty = E

  fun member (x, E) = false
    | member (x, T (_, l, v, r)) =
      if      Element.lt (x, v) then member (x, l)
      else if Element.lt (v, x) then member (x, r)
      else    true

  fun height E = 0
    | height (T (_, l, _, r)) = Int.max (height l, height r) + 1
  
  fun insert (x, t) =
    let fun balance (BLACK, T (RED, T (RED, a, x, b), y, c), z, d)   = T (RED, T (BLACK, a, x, b), y, T (BLACK, c, z, d))
          | balance (BLACK, T (RED, a, x, T (RED, b, y, c)), z, d)   = T (RED, T (BLACK, a, x, b), y, T (BLACK, c, z, d))
          | balance (BLACK, a, x, T (RED, T (RED, b, y, c), z, d))   = T (RED, T (BLACK, a, x, b), y, T (BLACK, c, z, d))
          | balance (BLACK, a, x, T (RED, b, y, T (RED, c, z, d)))   = T (RED, T (BLACK, a, x, b), y, T (BLACK, c, z, d))
          | balance body = T body
                           
        fun ins E = T (RED, E, x, E)
          | ins (s as T (color, l, v, r)) =
            if Element.lt (x, v) then balance (color, ins l, v, r)
            else if Element.lt (v, x) then balance (color, l, v, ins r)
            else s
        val T (_, l, v, r) = ins t  
    in  T (BLACK, l, v, r)
    end

  fun inOrder E = []
    | inOrder (T (_, l, v, r)) = (inOrder l) @ (v::(inOrder r))

  end

  structure IntOrdered : ORDERED =
  struct

  type T = int

  fun eq (a, b) = (a = b)
  fun lt (a, b) = (a < b)
  fun leq (a, b) = (a <= b)

  end

\end{lstlisting}

\begin{lstlisting}[
  caption={The red-black tree benchmark in SML: bench.sml},
  language=ml,
  basicstyle=\small,
  numbers=none,
  frame=single] 
  structure IntRBT = RedBlackTree(IntOrdered)
                 
  val _ =
    let val empty = IntRBT.empty
        val bigList = List.tabulate (1000001, fn x => x)
        val tree = List.foldl IntRBT.insert empty bigList
        fun printIntList xs = print (String.concatWith "\n"  (map Int.toString xs))
    in  print "Height of tree: ";
        print (Int.toString (IntRBT.height tree));
        print "\n";
        printIntList (IntRBT.inOrder tree)
    end
\end{lstlisting}

\newpage
The OCaml implementation of the benchmark consists of three files, ``rbt.mli'' (defines the interfaces),
``rbt.ml'' (implements the interfaces), and ``bench.ml'' (runs the benchmark). The program is compile using
the OCaml native code compiler (\lstinline{ocamlopt}) version 4.04.0, with the Spacetime profiling
tool enabled.

Note that although all implementations of the red-black tree benchmark are meant to be as self-contained as
possible, the OCaml implementation does have an external dependency, \lstinline{core}, due to the OCaml standard
library not providing efficient list utilities.

\begin{lstlisting}[
  caption={The red-black tree benchmark in OCaml: rbt.mli},
  language={[objective]caml},
  basicstyle=\small,
  numbers=none,
  frame=single]
  module type OrderedType = sig
    type t
    val compare : t -> t -> int
  end 
                                
  module type BinaryTree =
    sig
      type t
      type tree
      val empty : tree
      val member : t -> tree -> bool
      val height : tree -> int
      val insert : t -> tree -> tree
      val in_order : tree -> t list
    end
    
  module IntOrdered : sig
    type t = int
    val compare : t -> t -> int
  end

  module RedBlackTree :
    functor (Val : OrderedType) ->
      sig
        type t = Val.t
        type tree
        val empty : tree
        val member : t -> tree -> bool
        val height : tree -> int
        val insert : t -> tree -> tree
        val in_order : tree -> t list
      end
\end{lstlisting}

\begin{lstlisting}[
  caption={The red-black tree benchmark in OCaml: rbt.ml},
  language={[objective]caml},
  basicstyle=\small,
  numbers=none,
  frame=single]
  open Core

  module IntOrdered : (OrderedType with type t = int) = struct
    type t = int
    let compare a b =
      if a > b then 1
      else if a < b then -1
      else 0
  end

  module RedBlackTree(Val : OrderedType) : (BinaryTree with type t = Val.t) = struct

    type t = Val.t

    type color = Red | Black
    type tree =
        E
      | T of color * tree * t * tree

    let empty = E

    let rec member x t =
      match t with
      | E -> false
      | T (_, l, v, _) when Val.compare x v < 0 -> member x l
      | T (_, _, v, _) when Val.compare x v = 0 -> true
      | T (_, _, v, r) when Val.compare x v > 0 -> member x r

    let rec height = function
      | E -> 0
      | T (_, l, _, r) -> 1 + max (height l) (height r)

    let insert x t =
      let balance c l v r =
        match (c, l, v, r) with
        | (Black, T (Red, T (Red, a, x, b), y, c), z, d)
        | (Black, T (Red, a, x, T (Red, b, y, c)), z, d)
        | (Black, a, x, T (Red, T (Red, b, y, c), z, d))
        | (Black, a, x, T (Red, b, y, T (Red, c, z, d))) -> T (Red, T (Black, a, x, b), y, T (Black, c, z, d))
        | (c, l, v, r) -> T (c, l, v, r)
      in
      let rec ins = function
        | E -> T (Red, E, x, E)
        | T (color, l, v, r) as s->
           if Val.compare x v < 0 then balance color (ins l) v r
           else if Val.compare x v > 0 then balance color l v (ins r)
           else s
      in let T (_, l, v, r) = ins t in 
          T (Black, l, v, r)

    let rec in_order = function
      | E -> []
      | T (_, l, v, r) -> (in_order l) @ (v :: (in_order r))
  end
\end{lstlisting}

\begin{lstlisting}[
  caption={The red-black tree benchmark in OCaml: bench.ml},
  language={[objective]caml},
  basicstyle=\small,
  numbers=none,
  frame=single]
  open Core
  open Rbt

  module IntRBT = RedBlackTree(IntOrdered)
       
  let up_to (n: int) = 
    let rec up_to' n acc =
      match n with
      | 1 -> n::acc
      | _ -> up_to' (n-1) (n::acc)
    in up_to' n []
                            
  let (empty: IntRBT.tree) = IntRBT.empty
              
  let (bigList: int list) = up_to 1000000
                    
  let (tree: IntRBT.tree) = List.fold_left ~init:empty ~f:(Fn.flip IntRBT.insert) bigList
                          
  let print_int_list (xs: int list) = List.iter xs (Printf.printf "%d\n")

  let main _ =
    Printf.printf "Height of tree: %d\n" (IntRBT.height tree);
    print_int_list (IntRBT.in_order tree);
  ;;

  main ()
\end{lstlisting}

\newpage
The Scala implementation is contained in one file. The program is compiled using the Scala
compiler (\lstinline{scalac}) version 2.11.6, and executed on Oracle JDK 8 (version ``1.8.0\_161'').

\begin{lstlisting}[
  caption={The red-black tree benchmark in Scala},
  language=scala,
  basicstyle=\small,
  numbers=none,
  frame=single]
  sealed trait Color
  case object Red extends Color
  case object Black extends Color

  sealed abstract class Tree[+A <% Ordered[A]]
  final case object E extends Tree[Nothing]
  final case class T[A <% Ordered[A]](color: Color, left: Tree[A], elem: A, right: Tree[A]) extends Tree[A]


  object FunctionalRBT {
    def member[A <% Ordered[A]](x: A, t: Tree[A])(implicit or: Ordering[A]): Boolean = t match {
      case E => false
      case T(_, l, v, r) =>
        if (x < v) {
          member(x, l)
        } else if (x > v) {
          member(x, r)
        } else {
          true
        }
    }

    def insert[A <% Ordered[A]](x: A, t: Tree[A])(implicit or: Ordering[A]): Tree[A] = {
      def ins(s: Tree[A]): Tree[A] = s match {
        case E => T(Red, E, x, E)
        case T(color, l, v, r) =>
          if (x < v) {
            balance(color, ins(l), v, r)
          } else if (x > v) {
            balance(color, l, v, ins(r))
          } else {
            s
          }
      }

      val (l, v, r) = ins(t) match {
        case T(_, l, v, r) => (l, v, r)
      }

      T(Black, l, v, r)
    }

    def balance[A <% Ordered[A]](c: Color, l: Tree[A], v: A, r: Tree[A])
        (implicit or: Ordering[A]): Tree[A] = (c, l, v, r) match {
      case (Black, T(Red, T(Red, a, x, b), y, c), z, d) => T(Red, T(Black, a, x, b), y, T(Black, c, z, d))
      case (Black, T(Red, a, x, T(Red, b, y, c)), z, d) => T(Red, T(Black, a, x, b), y, T(Black, c, z, d))
      case (Black, a, x, T(Red, T(Red, b, y, c), z, d)) => T(Red, T(Black, a, x, b), y, T(Black, c, z, d))
      case (Black, a, x, T(Red, b, y, T(Red, c, z, d))) => T(Red, T(Black, a, x, b), y, T(Black, c, z, d))
      case (c, l, v, r) => T(c, l, v, r)
    }

    def inOrder[A <% Ordered[A]](t: Tree[A]): Unit = t match {
      case E =>;
      case T(_, l, v, r) => {
        inOrder(l)
        println(v)
        inOrder(r)
      }
    }

    def height[A <% Ordered[A]](t: Tree[A]): Int = t match {
      case E => 0
      case T(_, l, _, r) => {
        1 + math.max(height(l), height(r))
      }
    }

    def main(args: Array[String]): Unit = {
      import scala.language.implicitConversions
      import scala.language.postfixOps
      val nums = 1 to 1000000 toList
      val rbt = nums.foldLeft(E.asInstanceOf[Tree[Int]]) { (t, x) => insert(x, t) }
      println("Height of tree: " + height(rbt))
      inOrder(rbt)
    }
  }

\end{lstlisting}
