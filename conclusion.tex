Memory tracing is a useful and valuable dynamic analysis that can help programmers understand how they use heap memory in their programs, but it has
also been a very costly and particularly inefficient analysis to perform, making it impractical for most purposes. In light of this problem, this thesis
presents a tool, Elephant Tracks II, that is designed to make memory tracing useful again. To attain this goal, Elephant Tracks II uses an extensible, modular
design, and the JVM frontend for Elephant Tracks II particularly benefits from the modular architecture to improve the performance of memory tracing. In short,
the Elephant Tracks II project and this thesis's contributions are as following:
\begin{itemize}
\item providing an efficient, extensible and practical memory tracing framework;
\item developing an modular and effective architecture for memory tracing;
\item proposing new algorithms and techniques for running the Merlin algorithm offline;
\item uncovering new insights into the Merlin algorithm;
\item highlighting the importance of good design practices in research programming.
\end{itemize}

In the final analysis, the Elephant Tracks II project provides a new efficient, practical and extensible tool and architecture for memory tracing, as well as
new algorithms for and insights into running the Merlin algorithm offline. However, a little goes a long way: Elephant Tracks II makes it possible for program
analysis researchers and practitioners to analyze memory usage more easily, but we are still on our way to fully discover the value of memory tracing and of
Elephant Tracks II as a tool.
